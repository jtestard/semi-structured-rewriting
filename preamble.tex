

%------------------------------------------------------------------------------- 
% Math packages
%------------------------------------------------------------------------------- 

% The "amsmath" package provides advanced math extensions.
\usepackage{amsmath}

% The "amssymb" package adds new symbols to be used in math mode.
\usepackage{amssymb}

\usepackage{array}

% The "amsthm" package adds the "proof" environment and "theoremstyle" command.
% \usepackage{amsthm}

% The "appendix" package is for appendices
\usepackage[titletoc,toc,title]{appendix}

\usepackage{makecell}
\usepackage{array}
\newcolumntype{C}[1]{>{\centering\let\newline\\\arraybackslash\hspace{0pt}}m{#1}}
\newcolumntype{?}{!{\vrule width 1pt}}

% Comments
\newcommand{\jules}[1]{\reminder{\textcolor{blue}{\textbf{Jules: }}\textcolor{blue}{#1}}}
\newcommand{\yannis}[1]{\reminder{\textcolor{red}{\textbf{Yannis: }}\textcolor{red}{#1}}}

% For ceiling
\usepackage{mathtools}
\DeclarePairedDelimiter{\ceil}{\lceil}{\rceil}

%------------------------------------------------------------------------------- 
% Figure packages
%------------------------------------------------------------------------------- 

% The "fancyvrb" package provides advanced customization of verbatim environments, such as font families, numbering lines, box borders etc.
% \usepackage{fancyvrb}

% The "graphicx" package allows including external graphic files.
\usepackage{graphicx}
%\usepackage{circuitikz}

% The "subfig" package allows multiple sub-figures within a single figure, where sub-figures can be separately captioned and labeled, e.g. Figure % 1.2(a). This is a replacement for the older "subfigure" package.
\usepackage{subfig}

% HACK: The caption package (included by the subfig package) requires a counter for ACM's copyright box.
\newcounter{copyrightbox}

% The "float" package allows the "H" option for figures, which places a float % at a precise location.
\usepackage{float}

% The "caption" package allows captions for figures that are not actually in a floating environment (e.g. framed environment).
\usepackage{caption}

% The "framed" package creates framed regions that can break across pages.
\usepackage{framed}


% The "algorithm2e" package provides keywords for typesetting algorithms. The "noend" option disables the printing of the "end" keywords. Use "algomargin" to decrease the margins for all algorithms.

% Kevin: To resolve conflict of algorithm2e with other packages, a common problem with ACM template.
% See http://ergodicthoughts.blogspot.com/2009/06/latex-too-many-s-algorithm2e.html
\makeatletter
\newif\if@restonecol
\makeatother
\let\algorithm\relax
\let\endalgorithm\relax
\usepackage[noend,linesnumbered]{algorithm2e}
\setlength{\algomargin}{1.5em}

%------------------------------------------------------------------------------- 
% Layout packages
%------------------------------------------------------------------------------- 

% The "multirow" package allows table cells to span more than one row.
\usepackage{multirow}

% The "balance" package allows columns of the last page to be of equal height.
\usepackage{balance}

%------------------------------------------------------------------------------- 
% Whitespace packages
%------------------------------------------------------------------------------- 

% The "savetrees" package saves space on a page.
% \usepackage[all=normal,paragraphs=tight,floats=tight,bibnotes=tight]{savetrees}
% \usepackage[all=normal,paragraphs=tight,floats=tight,bibnotes=tight,bibliography=tight]{savetrees}

% The "setspace" package allows changing the inter-line spacing to be a multiple of the default line spacing.
%\usepackage{setspace}
%\setstretch{0.98}

% The "titlesec" package allows changing the whitespace around section headings.
% \usepackage[compact]{titlesec}

%------------------------------------------------------------------------------- 
% Misc packages
%------------------------------------------------------------------------------- 

% The "xcolor" package allows colored text and backgrounds.
\usepackage[table]{xcolor}

% The "soul" package allows highlighting.
\usepackage{soul}
\definecolor{gray}{rgb}{0.8,0.8,0.8}
\sethlcolor{gray}


% The "tocloft" packages allows generating custom lists that are similar to table of contents, list of figures etc.
% \usepackage[subfigure]{tocloft}

% The "hyperref" package allows creating hyperlinks. Note that it must be the last package loaded, and will automatically includes the "url" package.
\usepackage{hyperref}

% The "hypcap" package fixes "hyperref" so that hyperlinks go to the top of a float (as opposed to its caption).
\usepackage[all]{hypcap}

%------------------------------------------------------------------------------- 
% Whitespace
%------------------------------------------------------------------------------- 

% Adjust whitespace above and below captions
\addtolength{\abovecaptionskip}{-5pt}
\addtolength{\belowcaptionskip}{-9pt}

%------------------------------------------------------------------------------- 
% Macros
%------------------------------------------------------------------------------- 

% Define our own compact enumerate
\newenvironment{compact_enum}
{\setlength{\leftmargini}{1em}
\begin{enumerate}
  \setlength{\labelsep}{.3em} 
  \setlength{\itemsep}{.4em}
  \setlength{\parskip}{0pt}
  \setlength{\parsep}{0pt}}
{\end{enumerate}}

% Define our own compact itemize
\newenvironment{compact_item}
{\setlength{\leftmargini}{1em}
\begin{itemize}
  \setlength{\labelsep}{.3em} 
  \setlength{\itemsep}{.4em}
  \setlength{\parskip}{0pt}
  \setlength{\parsep}{0pt}}
{\end{itemize}}

% \newtheorem{theorem}{Theorem}[section]
% \newtheorem{lemma}[theorem]{Lemma}
% \newtheorem{proposition}[theorem]{Proposition}
% \newtheorem{corollary}[theorem]{Corollary}

% \newenvironment{proof}[1][Proof]{\begin{trivlist}
% \item[\hskip \labelsep {\bfseries #1}]}{\end{trivlist}}
% \newenvironment{definition}[1][Definition]{\begin{trivlist}
% \item[\hskip \labelsep {\bfseries #1}]}{\end{trivlist}}
\newenvironment{example}[1][Example]{\begin{trivlist}
\item[\hskip \labelsep {\bfseries #1}]}{\end{trivlist}}
% \newenvironment{remark}[1][Remark]{\begin{trivlist}
% \item[\hskip \labelsep {\bfseries #1}]}{\end{trivlist}}

% \newcommand{\qed}{\nobreak \ifvmode \relax \else
%       \ifdim\lastskip<1.5em \hskip-\lastskip
%       \hskip1.5em plus0em minus0.5em \fi \nobreak
%       \vrule height0.75em width0.5em depth0.25em\fi}


%------------------------------------------------------------------------------- 
% Database symbols
%------------------------------------------------------------------------------- 

\def\ojoin{\setbox0=\hbox{$\bowtie$}%
  \rule[-.02ex]{.25em}{.4pt}\llap{\rule[\ht0]{.25em}{.4pt}}}
\def\smojoin{\setbox0=\hbox{$\bowtie$}%
  \rule[-.068ex]{.20em}{.4pt}\llap{\rule[.742ex]{.20em}{.4pt}}}% small \ojoin for \smleftouterjoin
\def\leftouterjoin{\mathbin{\ojoin\mkern-5.8mu\bowtie}}
\def\smleftouterjoin{\mathbin{\smojoin\mkern-1.2mu\bowtie}}% small left outer join (for fractions)
\def\rightouterjoin{\mathbin{\bowtie\mkern-5.8mu\ojoin}}
\def\fullouterjoin{\mathbin{\ojoin\mkern-5.8mu\bowtie\mkern-5.8mu\ojoin}}
\def\semijoin{\mbox{$\mathrel{\raise1pt\hbox{\vrule height5pt depth0pt\hskip-1.5pt$>$\hskip -2.5pt$<$}}$}}
\def\antisemijoin{\overline{\semijoin}}
\def\innerjoin{\bowtie}
\def\crossproduct{\times}
%\def\topk{\operatorname{Topk}}
\def\topk{\lambda}
\def\ground{\operatorname{Ground}}
\def\scanop{\operatorname{Scan}}
\def\nav{\operatorname{Nav}}
\def\scannav{\operatorname{ScNa}}
\def\optscan{\operatorname{OptScan}}
\def\applyplan{\alpha}
\def\groupby{\gamma}
\def\partitionby{\chi}
\def\project{\pi}
\def\select{\sigma}
\def\sort{\tau}
\def\distinct{\delta}

\def\func{\lambda}


%------------------------------------------------------------------------------- 
% Cost model macros
%------------------------------------------------------------------------------- 

\newcommand{\pcost}[1]{\operatorname{cost}\left(#1\right)} % cost of a plan evaluation
\newcommand{\ncost}[1]{\operatorname{net}\left(#1\right)} % cost sending the result of #1 through the network
\newcommand{\ccost}[2]{\operatorname{cost}\left(#1,#2\right)} % cost of a plan context evaluation
\newcommand{\size}[1]{\##1} % size of a relation
\newcommand{\tpr}[1]{\operatorname{tpr}\left(#1\right)} % tuple-per-read
\newcommand{\fanout}[3]{\operatorname{fanout}\left(#1, #2, #3\right)} % fanout of #1 in #2 on #3
\newcommand{\absfanout}[2]{\operatorname{fanout}\left(#1, #2\right)} % number of tuplesin #1 that match condition #2

% Listing macros
\usepackage[noend,linesnumbered]{algorithm2e}
\usepackage{setspace} 
\usepackage{listings}
\definecolor{mygreen}{rgb}{0,0.6,0}
\definecolor{mygray}{rgb}{0.5,0.5,0.5}
\definecolor{mymauve}{rgb}{0.58,0,0.82}
\lstset{ %
  backgroundcolor=\color{white},   % choose the background color
  basicstyle=\footnotesize,        % size of fonts used for the code
  breaklines=true,                 % automatic line breaking only at whitespace
  captionpos=b,                    % sets the caption-position to bottom
  commentstyle=\color{mygreen},    % comment style
  escapeinside={\%*}{*)},          % if you want to add LaTeX within your code
  keywordstyle=\color{blue},       % keyword style
  stringstyle=\color{mymauve},     % string literal style
  numbers=left,
  numberstyle=\tiny\color{mygray},
}